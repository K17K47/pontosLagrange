\documentclass[10pt,twoside,a4paper,brazil]{abntex2}
\usepackage[utf8]{inputenc}
\usepackage[brazil]{babel}
\usepackage[T1]{fontenc}
\usepackage{amsmath}
\usepackage{amsfonts}
\usepackage{amssymb}
\usepackage{graphicx}
\usepackage{hyperref}
\usepackage{float}
\usepackage[num]{abntex2cite}
\usepackage{indentfirst}

\graphicspath{{./img/}}

\titulo{Pontos de Lagrange}

\autor{Benjamin G. de Figueiredo, Eduardo A. B. Oliveira, Henrique de A. Tórtura,\\Jonathas Q. R. Moraes, Lucas L. Costa, Octávio da Motta,\\Pedro A. S. de Alcântara, Wilson S. Martins} % TODO: Checar forma mais eficiente e esteticamente melhor de listar os contribuidores
\local{São Carlos, Brasil}
\data{2017}

\instituicao{
   Universidade de São Paulo -- USP
   \par
   Instituto de Física de São Carlos -- IFSC
}


\tipotrabalho{Artigo Científico}

\preambulo{Trabalho apresentado no curso de Mecânica Clássica, ministrado pelo professor Roberto Nicolau Onody, no Instituto de Física de São Carlos da Universidade de São Paulo, como parte do processo avaliativo da disciplina.}

\makeatletter

\hypersetup{
   pdftitle={\@title},
   pdfauthor={\@author},
   pdfsubject={\imprimirpreambulo},
   pdfkeywords={{Pontos de Lagrange}, {Mecânica Clássica}, {Estabilidade}, {Simulação Computacional}}, % TODO
   pdfcreator={LaTeX with abnTeX2},
   colorlinks=true,
   linkcolor=blue,
   citecolor=blue,
   urlcolor=blue
}

\makeatother

\begin{document}
   \imprimircapa
   
   \imprimirfolhaderosto   
   
   \tableofcontents

	 \newpage
	 
   \begin{resumo}
      Este trabalho visa expor e discutir questões acerca da concepção dos denominados pontos de Lagrange, que surgem naturalmente na análise de um problema de três corpos como uma extensão do problema de Kepler circular, apoiando-se inclusive nos textos precursores na obtenção dos mesmos, como publicações originais de Euler e Lagrange, sobre os três primeiros pontos e a inclusão dos dois últimos, respectivamente. Estrutura-se com o objetivo de desenvolver uma interpretação física e fenomenológica, partindo do problema restrito de dois corpos de onde obtivemos as funções horárias das componentes, as quais nos permitem escolher um referencial rotacional estacionário em relação a ambas. Extende-se o modelo adicionando um terceiro corpo consideravelmente menos massivo que os anteriores, e neste panorama de interação ternário, calculamos as coordenadas dos pontos de Lagrange através de métodos que permeiam o curso de graduação de Mecânica Clássica. Outrossim, uma investigação mais minunciosa dos pontos nos leva a conclusão de que dois deles são estáveis, equanto os demais são instáveis, exibindo seus expoentes de Lyapunov devido ao caráter caótico do sistema. Isso posto, um exemplo de aplicação é apresentado: o modelo de Roche para sistemas binários de estrelas. Nele, modela-se uma estrela hidrodinamicamente como um ponto contendo toda sua massa cercado por um envelope de massa nula. Além disso, apresentamos uma simulação computacional de particulas, escrita em C++11, baseada em um integrador numérico Runge-Kutta de quarta ordem para integração das equações newtonianas de movimento das partículas situadas nos pontos em que há interesse, a fim de que adquiríssemos informações relevantes para a caracterização dos pontos lagrangianos e constatações de suas estabilidades. O referido problema, por muito tempo, se apresentou como paradigma da física teórica, justificando, portanto, a realização dessa obra.

      \vspace{\onelineskip}
      \noindent
      \textbf{Palavras-chave}: Pontos de Lagrange, Mecânica Clássica, Estabilidade, Simulação Computacional
   \end{resumo}

   \chapter{Introdução}
      A Teoria Gravitacional proposta por Isaac Newton nos permite obter as equações de movimento para  dois corpos que se atraem mutuamente através da gravidade. Surge, no entanto, como decorrência dessa formalização uma complexa situação, na qual as ferramentas disponíveis tornam-se quase obsoletas, entraves caóticos provenientes da observação do chamado problema de três corpos. Diversos filósofos da natureza se debruçaram sobre a dinâmica envolvendo três corpos interagindo gravitacionalmente, buscando uma equação geral de movimento para as componentes do sistema. Cumpre ressaltar a importância particular de dois deles no que será abordado, avançando crucialmente a compreensão humana acerca dos referido problema: Leonhard Paul Euler, com sua primazia, e Joseph-Louis Lagrange, com o esclarecimento do que havia sido deixado em aberto pelo primeiro.
   
Esse avanço foi alcançado lançando mão de certas considerações. Na década de 60 do século XVIII, Euler modelou um sistema trinário unidimensional* e, com isso, indicou a existência de configurações estacionárias. Em suma, dados dois corpos, existem três pontos na reta que passa pelo centro de massa de ambos nos quais o posicionamento do terceiro corpo faz com que a distribuição relativa entre os três permaneça a mesma. Por conseguinte, seu aluno Lagrange, em 1772, dando prosseguimento as descobertas sobre equilíbrio no problema de três corpos, ao trabalhar em um sistema tridimensional, executando a maioria dos cálculos considerando apenas dois corpos e adicionando em seguida o terceiro, menos massivo*, obteve as posições de um quarto e um quinto ponto com a mesma propriedade estacionária dos anteriores. Dos cinco pontos citados, os três primeiros, descobertos por Euler, correspondem a pontos de equilibrio instável do sistema. Os demais, publicados por Lagrange, são estáveis. A todos eles dá-se o nome \textit{pontos de Lagrange}. A figura 1 retrata os pontos marcados como L1, L2, L3, L4 e L5.

\begin{figure}[!h]
\centering
\includegraphics{PL.gif}
\caption{Curvas equipotenciais e pontos de Lagrange. Retirado de CITAR KEPLER DE SOUZA.}
\end{figure}
   
Em decorrência da estabilidade dos pontos, é natural o acúmulo de matéria nas regiões em volta de L4 e L5. Um exemplo claro disso, são os asteróides troianos de Júpiter: uma família de corpos celestes que se dispoem nos pontos de equilíbrio estável do sistema Sol-Júpiter. Os outros pontos, apesar de instáveis, são largamente utilizados por instrumentos de pesquisa espacial. O SOHO (Solar and Heliospheric Observatory -- Observatório Solar e Heliosférico)** se mantem numa órbita em volta do ponto L1 do sistema Sol-Terra, permitindo à sonda uma visão initerrupta do Sol, sem sofrer com eclipses causados pelo nosso planeta. Numa órbita em torno do ponto L2, há o WMAP (Wilkinson Microwave Anisotropy Probe -- Sonda de Anisotropia de Microondas Wilkinson)**, cuja localização lhe fornece uma visão clara do espaço profundo.

%   *: De motu rectilineo trium corporum se mutuo attrahentium (http://eulerarchive.maa.org//docs/originals/E327.pdf), Euler;
%      Essai sur le problème des trois corps (http://gallica.bnf.fr/ark:/12148/bpt6k229225j/f231.image.r=Oeuvres+de+Lagrange.langFR), Lagrange.

%   **: SOHO's Orbit (https://sohowww.nascom.nasa.gov/about/orbit.html);
%       WMAP Trajectory and Orbit (https://map.gsfc.nasa.gov/mission/observatory_orbit.html).


      
   \chapter{Teoria}
      \section{Gravitação e o problema de n-corpos}

% TODO Descrição breve da gravitação na mecânica clássica, introduzindo o potencial efetivo, sua interpretação e propriedades relevantes.
% Problema de n-corpos na gravitação, casos reduzidos, prova de existência de pontos estacionários pro potencial efetivo newtoniano


\section{Dedução dos pontos de Lagrange}

% TODO Dedução os pontos L1, L2, L3, L4 e L5. Estabilidade dos pontos de Lagrange, valor de Gascheau.
% https://map.gsfc.nasa.gov/ContentMedia/lagrange.pdf

\section{Generalizações}
% TODO Discutir resultados recentes que generalizam a noção de ponto de Lagrange.
% http://iopscience.iop.org/article/10.1086/513736/meta
% http://iopscience.iop.org/article/10.1086/423214/meta

     
   \chapter{Simulação Computacional}
         Para melhor visualização dos fenômenos previamente expostos, foram elaboradas simulações pertinentes para análise de sistemas de mecânica celeste \cite{sashalag, sashaeng} -- utilizando a linguagem C++. Trata-se de um sistema fictício de duas partículas de massas $M_1 = 10^{12}$ e $M_2 = 10^{10}$ unidades de massa e distância inicial da ordem $10^6$ unidades de distância. Uma terceira partícula de massa $10^2$ unidades de massa é então inserida em cada um dos pontos de Lagrange por vez, com o intuito de verificar a evolução temporal do sistema. A simulação é iniciada com $M_1$ e $M_2$ no eixo horizontal com respectivas velocidades $(0,0)$ e $(0, 10^6)$. A velocidade inicial da terceira partícula foi apropriadamente escolhida para um melhor acoplamento orbital em cada ponto de Lagrange, a saber: $(0,1,083942 \cdot 10^6)$ para L1, $(0,9,329572 \cdot 10^5)$ para L2, $(0,-9,978991 \cdot 10^5)$ para L3, $(-8,68161\cdot 10^5, 5,01233 10^5)$ para L4 e $(8,68161\cdot 10^5, 5,01233 \cdot 10^5)$ para L5. Os resultados, dispostos nas imagens a seguir, variam de acordo com os pontos analisados em cada uma das simulações apresentadas nas respectivas imagens, as partículas de massa $10^2$ são grafadas por $P_{N}$ com seus respectivos pontos de Lagrange $L_{N}$ marcados no instante inicial, com $N = 1,..., 5$. 
   
\begin{figure}[h]
\begin{minipage}{0.5\textwidth}
\centering
\includegraphics[width = 3 in]{./sim/L1-final.png}
\caption{Trajetória da partícula 1}
\end{minipage}
\begin{minipage}{0.5\textwidth}
\centering
\includegraphics[width = 3 in]{./sim/L2-final.png}
\caption{Trajetória da partícula 2}
\end{minipage}
\end{figure}

\begin{figure}[h]
\begin{minipage}{0.5\textwidth}
\centering
\includegraphics[width = 3 in]{./sim/L3-final.png}
\caption{Trajetória da partícula 3}
\end{minipage}
\begin{minipage}{0.5\textwidth}
\centering
\includegraphics[width = 3 in]{./sim/L4-final.png}
\caption{Trajetória da partícula 4}
\end{minipage}
\end{figure}

\begin{figure}[H]
\centering
\includegraphics[width = 3 in]{./sim/L5-final.png}
\caption{Trajetória da partícula 5}
\end{figure}

   A instabilidade de L1, L2 e L3 é evidenciada pela trajetória descrita pelo terceiro corpo. Ao ser acelerado pelo efeito estilingue, ele se desvia de sua órbita. Interessante notar que a órbita de P3 é consideravelmente mais estável do que as de P1 e P2, comportamento esperado pelo autovalor descrito no capítulo 2. Por outro lado, em L4 e L5, podemos observar órbitas bem comportadas, o que configura a estabilidade dos pontos.


   \chapter{Conclusão}
         O presente trabalho objetivava apresentar o conceito de Pontos de Lagrange, bem como explicitar a teoria envolvida para a compreensão desse fenômeno, amplamente discutida no contexto da Mecânica Clássica. Para tal, foram utilizadas diversas ferramentas, como simulações computacionais e formalizações matemáticas, discutidas no decorrer dos textos, a fim de evidenciar as características, analisar as consequências, e estudar os mecanismos pelos quais são descritas interações entre três corpos, sob determinadas condições. Os problemas que circunscrevem esse fenômeno, por muito tempo, se apresentaram como  paradígmas da física teórica, justificando, portanto, a realização dessa obra.


   \bibliography{bib/general.bib}
\end{document}
