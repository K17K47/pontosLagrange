\section{Gravitação e o problema de n-corpos}

% TODO Descrição breve da gravitação na mecânica clássica, introduzindo o potencial efetivo, sua interpretação e propriedades relevantes.
% Problema de n-corpos na gravitação, casos reduzidos, prova de existência de pontos estacionários pro potencial efetivo newtoniano
% https://arxiv.org/abs/astro-ph/0503600
% http://web.physics.ucsb.edu/~mccann/notes/Effective_Potential.pdf

Sistemas gravitacionais de mais de dois corpos são, em geral, impossíveis de se resolver analiticamente. Num sistema de $n$ partículas interagindo entre si exclusivamente gravitacionalmente, a equação de movimento da $i$-ésima partícula é 

\begin{equation}
    \ddot{\mathbf{r}}_i=-\sum_{\substack{j=1\\j \neq i}}^{n} \frac{Gm_j(\mathbf{r}_j - \mathbf{r}_i)}{||\mathbf{r}_j - \mathbf{r}_i||^3}.
\end{equation} 

Trata-se de uma situação consideravelmente mais complexa que a de poucos corpos, uma vez que não há mais conservação de momento angular, a possibilidade de colisões gera singularidades nas soluções e o desacoplamento das equações não é possível. Apesar disso, é possível analisar e resolver sistemas de $n$-corpos, especialmente na presença de restrições e aproximações adequadas.

Com efeito, num sistema coplanar descrito por um referencial baricêntrico que gira com velocidade angular constante $\omega$, definindo o potencial efetivo por unidade de massa

\begin{equation}
    U_{eff}(\mathbf{r}) = -\sum_{i = 1}^n \frac{Gm_i}{||\mathbf{r} - \mathbf{r}_i||} - \frac{1}{2}\mathbf{r}^2\omega^2
\end{equation}

e escolhendo um ponto no espaço $\mathbf{r_0} \neq \mathbf{r}_i$, observamos que o conjunto

\begin{align} 
    \Lambda &= \{\mathbf{r} | U_{eff}(\mathbf{r}) \geq U_{eff}(\mathbf{r_0}) - \epsilon\} && \epsilon > 0
\end{align}

é fechado e limitado, uma vez que o potencial tende a descrescer. Por virtude da continuidade dessa função no compacto $\Lambda$, ela atinge nele um máximo $\mathbf{r_M}$. Esse máximo está necessariamente contido no interior de $\Lambda$, já que $U_{eff}(\mathbf{r_M}) \geq U_{eff}(\mathbf{r_0}$, e pela diferenciabilidade de $U_{eff}$ o ponto é estacionário: um objeto de massa negligível nesse ponto realizará movimento circular uniforme ao redor do centro de massa do sistema. Veremos mais adiante que pontos de maximização como esse são, num sentido não muito trivial, atrativos.

Para fins práticos, entretanto, existência é insatisfatória. Isso, dentre outros fatores, torna a descoberta por Euler e Lagrange dos cinco pontos de Lagrange notável, uma vez que seu cálculo é razoavelmente simples e, por serem estacionários para o potencial de uma órbita de Kepler, encontram grande aplicabilidade em situações reais.

\section{Dedução dos pontos de Lagrange}

% TODO Dedução os pontos L1, L2, L3, L4 e L5. Estabilidade dos pontos de Lagrange, valor de Gascheau.
% https://map.gsfc.nasa.gov/ContentMedia/lagrange.pdf

\section{Generalizações}
% TODO Discutir resultados recentes que generalizam a noção de ponto de Lagrange.
% http://iopscience.iop.org/article/10.1086/513736/meta
% http://iopscience.iop.org/article/10.1086/423214/meta
