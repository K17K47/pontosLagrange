\section{Gravitação e o problema de n-corpos}

Dois corpos massivos ocupando posições diferentes no espaço sofrem uma atração proporcional ao produto de suas massas e ao inverso do quadrado de sua distância mútua no sentido que os une. Isso é descrito pelo potencial,

\begin{equation}
    \label{eq:gravitacaouniversal}
    U(\mathbf{r}_1 - \mathbf{r}_2) = -\frac{Gm_1m_2}{||\mathbf{r}_1 - \mathbf{r}_2||}
\end{equation}

Mas o espaço é homogêneo, e o movimento do sistema pode depender apenas das posições relativas dos objetos. Isso motiva que se defina

\begin{align}
    \mathbf{R} &= \mathbf{r}_1 - \mathbf{r}_2 && \mu = \frac{m_1m_2}{m_1 + m_2}
\end{align}

e as equações de movimento se reduzem a

\begin{equation}
    \mu \ddot{\mathbf{R}} = -\frac{\partial U}{\partial \mathbf{R}}
\end{equation}

e como

\begin{align}
    \mathbf{r}_1 &= \mathbf{r}_{CM} + \frac{m_2}{m_1 + m_2}\mathbf{R} && \mathbf{r}_2 = \mathbf{r}_{CM} - \frac{m_1}{m_1 + m_2}\mathbf{R}
\end{align}

conclui-se que num referencial baricêntrico o problema de dois corpos é simplesmente o problema de força central, que é resolvido, pela mudança de variável $u = \frac{1}{r}$ num sistema de coordenadas polar perpendicular ao momento angular $\mathbf{L}$ (que é conservado), através da equação de Binet,

\begin{equation}
    \label{eq:binet}
    \frac{d^2u}{d\theta^2} + u = -\frac{\mu}{\mathbf{L}^2}F\left(\frac{1}{u}\right).
\end{equation}

Assim, obtem-se a trajetória e, com algum esforço, a evolução temporal do sistema. Sistemas de dois corpos ou de um corpo (caso limite para um dos corpos muito mais massivo que o outro) admitem, em situações de energia negativa, órbitas elípticas

\begin{equation}
    r(\theta) = \frac{a(1 - e^2)}{1 + e\cos (\theta - \theta_0)}.
\end{equation}

Sistemas gravitacionais de mais de dois corpos são, por outro lado, impossíveis de se resolver analiticamente no caso geral. Num sistema de $n$ partículas interagindo entre si exclusivamente gravitacionalmente, a equação de movimento da $i$-ésima partícula é 

\begin{equation}
    \ddot{\mathbf{r}}_i=-\sum_{\substack{j=1\\j \neq i}}^{n} \frac{Gm_j(\mathbf{r}_j - \mathbf{r}_i)}{||\mathbf{r}_j - \mathbf{r}_i||^3}.
\end{equation} 

Trata-se de uma situação consideravelmente mais complexa que a de poucos corpos, uma vez que não há mais conservação de momento angular de cada partícula, a possibilidade de colisões gera singularidades nas soluções e o desacoplamento das equações não é possível. Apesar disso, é possível analisar e resolver sistemas de $n$-corpos, especialmente na presença de restrições e aproximações adequadas.

Os pontos de Lagrange emergem naturalmente na análise de um problema de três corpos como uma extensão do problema de Kepler circular: Sendo $R$ a distância entre dois objetos de massas $m_1$ e $m_2$, tomamos um referencial baricêntrico que gira com velocidade angular constante $\omega$ igual à orbital

\begin{equation}
    \omega^2 = \frac{G(m_1 + m_2)}{R^3}
\end{equation}

Caso um corpo estacionário de massa $m$ seja inserido nesse sistema, preservando coplanaridade, observar-se-á agindo sobre ele uma força centrífuga (e, caso se movimentasse, também uma força de Coriolis. Garantidamente não há força de Euler nesse tratamento), de forma que o potencial efetivo se torna

\begin{equation}
    U_{ef}(\mathbf{r}) = - \frac{Gmm_i}{||\mathbf{r} - \mathbf{r}_1||} - \frac{Gmm_i}{||\mathbf{r} - \mathbf{r}_2||} - \frac{1}{2}\mathbf{r}^2\omega^2
\end{equation}

e escolhendo um ponto no espaço $\mathbf{r_0} \neq \mathbf{r}_i$, observamos que o conjunto

\begin{align} 
    \Lambda &= \{\mathbf{r} | U_{ef}(\mathbf{r}) \geq U_{ef}(\mathbf{r_0}) - \epsilon\}
\end{align}

é, para $\epsilon > 0$, fechado e, como $U_{eff}$ tende a $-\infty$ pra $||\mathbf{r}||$ grande, limitado e portanto compacto. Assim, a função atinge nele um máximo $\mathbf{r_M}$. $\mathbf{r_M}$ está necessariamente contido no interior de $\Lambda$, já que $U_{ef}(\mathbf{r_M}) \geq U_{ef}(\mathbf{r_0})$. Pela diferenciabilidade de $U_{ef}$ o ponto é estacionário: um objeto de massa negligível nesse ponto realizará movimento circular uniforme ao redor do centro de massa do sistema.

Para fins práticos, entretanto, existência é insatisfatória, mas motiva o cálculo desses pontos, dentre outros pontos estacionários do potencial efetivo, que serão justamente os pontos de Lagrange.

\section{Problema restrito de 2 corpos}

Nas coordenadas do centro de massa, podemos escrever:

\begin{align}
    \mathbf{r}_1 &=  \frac{m_2}{m_1 + m_2}\mathbf{R} && \mathbf{r}_2 = - \frac{m_1}{m_1 + m_2}\mathbf{R}
\end{align}

E, analisando a força gravitacional sobre um dos corpos:

\begin{equation}
\mathbf{F} = m_1\mathbf{\ddot{r}}_1 = \mu\mathbf{\ddot{R}} = -\frac{Gm_1m_2}{R^3}\mathbf{R} \label{dia}
\end{equation}

E note que ocorre exatamente o mesmo para o corpo 2. Podemos escrever a (\ref{dia}) como:

\begin{equation}
\mathbf{\ddot{R}} + \frac{G(m_1+m_2)}{R^3}\mathbf{{R}} = 0 \label{dif}
\end{equation}

E, uma vez resolvida a (\ref{dif}), obtêm-se as funções horárias de ambas as partículas. Nos restringiremos aqui ao caso em que a distância entre os 2 corpos é constante. Nesse caso, a equação se torna simplesmente uma equação de oscilador harmônico em cada uma das coordenadas, e $\mathbf{R}$
realiza um movimento circular uniforme cuja frequência é dada por:

\begin{equation}
\omega^2 = \frac{G(m_1+m_2)}{R^3} \label{kepi}
\end{equation}

\section{Dedução dos pontos de Lagrange}

Interessa-nos agora analisar uma extensão do problema anterior, adicionando no sistema um terceiro corpo, de massa $m<<m_1,m_2$, de maneira que ele não perturbe a órbita original nem desloque o centro de massa do sistema. 

\begin{figure}[H]
\centering
\includegraphics[scale = 0.5]{3_corpos.png}
\caption{Problema restrito de 3 corpos.}
\end{figure}


Temos que as forças atuantes sobre ele são:

\begin{align}
\mathbf{F_1} = -\frac{Gm_1m}{|\mathbf{r}-\mathbf{r_1}|^3}(\mathbf{r}-\mathbf{r_1}) && \mathbf{F_2} = -\frac{Gm_2m}{|\mathbf{r}-\mathbf{r_2}|^3}(\mathbf{r}-\mathbf{r_2})
\end{align}

Nessas condições, chamamos de pontos de Lagrange os pontos que permitem uma órbita estacionária para o terceiro corpo, mantendo constantes as distâncias entre os três corpos.

Como $\mathbf{r_1}$ e $\mathbf{r_2}$ são funções do tempo, torna-se mais conveniente tratar do problema num referencial baricêntrico que gira com velocidade angular constante $\omega$. Nesse caso, devemos levar em consideração as forças de inércia:

\begin{equation}
\mathbf{F_{IN}} = -m\boldsymbol{\omega} \times (\boldsymbol{\omega} \times \mathbf{r}) -2m(\mathbf{\omega} \times \mathbf{\dot{r}}) 
\end{equation}

Onde o primeiro termo corresponde à força centrífuga e o segundo, à força de Coriolis. Note que nesse referencial, os pontos de Lagrange correspondem a pontos de equilíbrio estático, nos quais a força resultante (incluindo as forças de inércia) é nula. Para uma solução estática, devemos, é claro, ter que $\mathbf{\dot{r}}=0$. Escolhendo coordenadas cartesianas convenientemente alinhadas, temos:

\begin{eqnarray}
\boldsymbol{\omega} & = & \omega\mathbf{\hat{k}} \\
\mathbf{r} \; & = & x\mathbf{\hat{i}} + y\mathbf{\hat{j}} \\
\mathbf{r_1} & = & -\alpha R\mathbf{\hat{i}} \\
\mathbf{r_2} & = & \beta R\mathbf{\hat{i}} \\
\end{eqnarray}

Onde,

\begin{align}
\alpha = \frac{-m_2}{m_1+m_2} && \beta = \frac{m_1}{m_1+m_2}
\end{align}

Escrevemos então a força resultante sobre o terceiro corpo:


\begin{eqnarray}
& \mathbf{F_{IN}} + \mathbf{F_1} + \mathbf{F_2} \\
& = m\left[\omega^2\mathbf{r} -G(m_1+m_2)\left(\frac{\beta(\mathbf{r}-\mathbf{r_1})}{|\mathbf{r}-\mathbf{r_1}|^3} + \frac{\alpha(\mathbf{r}-\mathbf{r_2})}{|\mathbf{r}-\mathbf{r_2}|^3}\right) \right] \\
& = m\omega^2 \left[\left(x - \frac{\beta(x+\alpha R) R^3)}{((x+\alpha R)^2 + y^2)^{3/2}} - \frac{\alpha(x-\beta R) R^3)}{((x-\beta R)^2 + y^2)^{3/2}}\right)\mathbf{\hat{i}} + \left(y - \frac{y R^3}{((x+\alpha R)^2 + y^2)^{3/2}} - \frac{y R^3}{((x-\beta R)^2 + y^2)^{3/2}}\right)\mathbf{\hat{j}}\right] 
\end{eqnarray}

Onde utilizamos a (\ref{kepi}). queremos os pares ordenados (x,y) onde a força se anula. Vemos que não é trivial resolver esse sistema diretamente. Entretanto, podemos induzir as respostas levando em consideração suas simetrias.

Considerando a simetria por reflexão no eixo x, temos que a função tem que se anular na reta $y=0$. Definimos:

\begin{equation}
x \equiv (u+\beta)R
\end{equation}

De maneira que u mede a distância do corpo até $m_2$ em unidades de R. A equação resulta em:

\begin{eqnarray}
R(u+\beta) - \frac{\beta R^4(u+\beta+\alpha)}{R^3(u+\beta+\alpha)^3}- \frac{\alpha R^4u}{R^3u} & = & 0 \\
R\left[\frac{u^2(u+\beta+\alpha)^2(u+\beta)-\beta u^2 - \alpha(u+\beta+\alpha)^2}{u^2(u+\beta+\alpha)^2}\right] & = & 0 \label{poliu}
\end{eqnarray}

E o problema se passa a ser encontrar as raízes do polinômio de grau 5 no numerador da (\ref{poliu}), onde não há uma solução fechada. Nesse caso, consideraremos a hipótese $\alpha << 1$, de maneira que só consideraremos termos de primeira ordem. Nesse caso, podemos reescrever o polinômio como:

\begin{equation}
u^2((1-s_1)+3u+3u^2+u^3) = \alpha(s_0 +2s_0u+(1+s_0-s_1)u^2+2u^3+u^4) \label{polia}
\end{equation}

Onde $s_0$ corresponde ao sinal de $u$ e $s_1$, ao sinal de $u+1$. Os 3 casos possíveis para o par ordenado $(s_0,s_1)$ são $(-1,-1), (-1,1), (1,1)$, que correspondem, respectivamente, a posições à esquerda de $m_1$, entre $m_1$ e $m_2$ e à direita de $m_2$. Claramente o par $(1,-1)$ não pode ocorrer. Em cada caso, a equação (\ref{polia}) tem uma raiz real, que correspondem aos pontos.

\begin{eqnarray}
L1: & R\left[1-\left(\dfrac{\alpha}{3}\right)^{1/3}\right]\mathbf{\hat{i}} \\
L2: & R\left[1+\left(\dfrac{\alpha}{3}\right)^{1/3}\right]\mathbf{\hat{i}} \\
L3: & -R\left[1 + \dfrac{5}{12}\alpha \right]\mathbf{\hat{i}}
\end{eqnarray}

Os pontos L4 e L5 não estão sobre o eixo, mas podem ser determinados procurando por pontos onde a componente radial da força gravitacional é anulada pela força centrífuga. Isso sugere uma decomposição da força total em termos paralelos e perpendiculares à posição, ou, em termos vetoriais, projeções de $\mathbf{F_R}$ nas direções $(x, y)$ e $(-y, x)$. A projeção perpendicular

\begin{align}
    F_R^\perp &= \alpha\beta y \omega^2R^3\left( \frac{1}{((x-R\beta)^2+y^2)^{3/2}} - \frac{1}{((x + R\alpha)^2+y^2)^{3/2}} \right)
\end{align}

nos dá, quando anulada, e portanto igualando os termos dos denominadores,

\begin{equation}
    x = \frac{R\beta - R\alpha}{2} = \frac{r_1 - r_2}{2}
\end{equation}

ou seja, a coordenada $x$ dos pontos que procuramos estão no ponto médio entre os objetos. Isso nos dá a projeção paralela

\begin{equation}
    F_R^\parallel = \omega^2\frac{x^2 + y^2}{R}\left( \frac{1}{R^3} - \frac{1}{((x-R\beta)^2 + y^2)^{3/2}}\right),
\end{equation}

que anulada equivale a

\begin{equation}
    R^2=(x-R\beta)^2+y^2
\end{equation}

e torna-se claro que, como os pontos simultaneamente distam R de um dos (e portanto ambos os) corpos e os equidistam, localizam-se nos vértices dos dois triângulos equiláteros determinados pela aresta que liga os corpos no espaço, ou, escrito explicitamente,

\begin{align}
    L4: & \quad \frac{R}{2}\left(\frac{m_1 - m_2}{m_1 + m_2}\right)\mathbf{\hat{i}} +\frac{\sqrt{3}}{2}R\;\mathbf{\hat{j}} \\
    L5: & \quad \frac{R}{2}\left(\frac{m_1 - m_2}{m_1 + m_2}\right)\mathbf{\hat{i}}  -\frac{\sqrt{3}}{2}R\;\mathbf{\hat{j}}
\end{align}

\section{Estabilidade}

Podemos associar à força resultante um potencial generalizado, tal que:

\begin{equation}
\mathbf{F_R} = -\boldsymbol{\nabla} U + \frac{d}{dt}(\boldsymbol{\nabla _v}U)
\end{equation}

Onde os pontos de equilíbrio encontrados correspondem a pontos onde a variação do potencial é estacionária. Analisaremos agora o comportamento da órbita para pequenos deslocamentos dos pontos de equilíbrio. Considerando o $i$-ésimo ponto de Lagrange:

\begin{eqnarray}
x = x_i + \delta x, & \delta v_x \\
y = y_i + \delta y, & \delta v_y
\end{eqnarray}

descrevemos a equação de movimento linearizada como

\begin{equation}
\frac{d}{dt}
  \begin{bmatrix}
    \delta x \\
    \delta y \\
    \delta v_x \\
    \delta v_y
 \end{bmatrix}
=
 \begin{bmatrix}
   0 & 0 & 1 & 0 \\
   0 & 0 & 0 & 1 \\
   \dfrac{1}{m} \dfrac{\partial ^2U}{\partial x^2} & \dfrac{1}{m} \dfrac{\partial ^2U}{\partial x\partial y} & 0 & 2\omega \\
   \dfrac{1}{m} \dfrac{\partial ^2U}{\partial y \partial x} & \dfrac{1}{m} \dfrac{\partial ^2U}{\partial y^2} & 2\omega & 0
 \end{bmatrix}
   \begin{bmatrix}
    \delta x \\
    \delta y \\
    \delta v_x \\
    \delta v_y
 \end{bmatrix}
\end{equation}

e, cientes de que para equações diferenciais da forma

\begin{equation}
    \dot{\mathbf{x}}(t) = A\mathbf{x}(t)
\end{equation}

a solução é $\mathbf{x}(t) = e^{At}$, buscamos diagonalizar a matriz de evolução. Se para algum autovalor $\lambda_i$ associado às velocidades tivermos que $\Re(\lambda_i) > 0$, a exponencial da matriz na base de autovetores, que nada mais será que $\text{diag}_i(e^{\lambda_i t})$, cresce exponencialmente na i-ésima coordenada, e portanto o ponto de Lagrange associado é instável. Caso contrário, o modelo linear é estável, pois em todas as direções a matriz ou tende a zero ou oscila.

\subsection{Pontos L1 e L2}

Nos pontos L1 e L2, temos:

\begin{equation}
\dfrac{\partial ^2U}{\partial x^2} = \mp 9\omega^2 \;, \qquad \qquad \dfrac{\partial ^2}{\partial y^2} = \pm 3\omega^2 \;, \qquad \qquad \dfrac{\partial ^2U}{\partial x\partial y} = \dfrac{\partial ^2U}{\partial y\partial x} = 0
\end{equation}

\vspace{10px}

Os autovalores da matriz então resultam em:

\vspace{-13px}

\begin{align}
\lambda_{\pm} = \pm \omega \sqrt{1+2\sqrt{7}} && \sigma_{\pm} = \pm i\omega \sqrt{2\sqrt{7}-1}
\end{align}

O autovalor positivo $\lambda_+$ indica a presença de uma solução que diverge exponencialmente dos pontos de equilíbrio, o que significa que uma órbita nos pontos L1 e L2 é instável. O tempo característico da divergência é:

\begin{equation}
\tau = \frac{1}{\lambda_+} \approx \frac{2}{5\omega}
\end{equation}

\subsection{Ponto L3}

Já no ponto L3, as derivadas resultam:

\begin{equation}
\dfrac{\partial ^2U}{\partial x^2} = -3\omega^2 \;, \qquad \qquad \dfrac{\partial ^2}{\partial y^2} = \frac{7m_2}{8m_1}\omega^2 \;, \qquad \qquad \dfrac{\partial ^2U}{\partial x\partial y} = \dfrac{\partial ^2U}{\partial y\partial x} = 0
\end{equation}

E os autovalores equivalentes são:

\begin{align}
\lambda_{\pm} = \pm \omega \left(\dfrac{3m_1}{8m_2}\right)^{1/2} && \sigma_{\pm} = \pm i\sqrt{7}
\end{align}

Novamente, temos um autovalor real e positivo, de maneira que L3 também é instável.

\subsection{Pontos L4 e L5}

Esses dois últimos pontos são um caso curioso no que se refere à sua estabilidade. Ambos são pontos de máximo do potencial $U$, o que normalmente indicaria instabilidade. O que ocorre de fato é que, para uma massa em repouso na vizinhança de um desses pontos, ela inicialmente se afasta do ponto; entretanto, ao ganhar velocidade, ela passa a estar sujeita à força de Coriolis, que a faz realizar uma órbita em torno do ponto de Lagrange.

Para esses pontos, temos os valores:

\begin{equation}
\dfrac{\partial ^2U}{\partial x^2} = \dfrac{3}{4}\omega^2 \;, \qquad \qquad \dfrac{\partial ^2}{\partial y^2} = \dfrac{9}{4}\omega^2 \;, \qquad \qquad \dfrac{\partial ^2U}{\partial x\partial y} = \dfrac{\partial ^2U}{\partial y\partial x} = \dfrac{3\sqrt{3}}{4}\gamma \omega^2
\end{equation}

\vspace{20px}

Onde definimos: 

\vspace{-35px}

\begin{equation*}
\!\!\!\!\!\!\!\!\!\!\!\!\!\!\!\!\!\!\!\!\!\!\!\!\!\!\!\!\!\!\!\!\!\!\!\!\!\!\!\!\!\!\!\!\!\!\!\!\!\!\!\!\!\!\!\!\!\!\!\!\!\!\!\!\!\!\!\!\!\!\!\!\!\!\!\!\!\!\!\!\!\!\!\!\!\!\!\!\!\!\!\!\!\!\!\!\!\!\!\!\!\!\!\!\!\!\!\!\!\!\!\!\!\!\!\!\!\!\!\!\!\!\!\!\!\! \gamma = \dfrac{m_1-m_2}{m_1+m_2}
\end{equation*}

E, os autovalores correspondentes são:

\begin{align}
\lambda_{\pm} = \pm \frac{i}{2} \omega \sqrt{2-\sqrt{27\gamma^2-23}} && \sigma_{\pm} = \pm \frac{i}{2} \omega \sqrt{2 +\sqrt{27\gamma^2 - 23}}
\end{align}

Os pontos serão estáveis se os autovalores forem puramente imaginários. Para isso devem ser satisfeitas as condições:

\begin{align}
\gamma^2 \geq \frac{23}{27} && \text{e} && \sqrt{27\gamma^2 -23} \leq 2
\end{align}

A primeira condição é automaticamente satisfeita por considerarmos $\alpha << 1$ na seção (2.3). A segunda condição resulta em:

\begin{equation}
m_1 \geq 25m_2\left(\dfrac{1+\sqrt{1-4/625}}{2}\right)
\end{equation}

\section{Aplicações ao modelo de Roche}

\subsection{Modelo de Roche}

No estudo de sistemas binários de estrelas, um dos modelos fundamentais da estrela é o de Roche. Nele, modela-se a estrela hidrodinamicamente como um ponto contendo toda a sua massa cercado por um envelope de massa nula. O potencial generalizado é expresso, para uma das estrelas, num referencial baricêntrico rotacionando uniformemente com eixo $x$ no sentido que une os centros das duas estrelas, por

\begin{equation}
    U_{eff} = -\frac{Gm_1}{||\mathbf{r}_1||}-\frac{Gm_2}{||\mathbf{r}_2||}-\frac{1}{2}\Omega_1^2(x^2 + y^2)+\frac{Gm_2x}{||\mathbf{r}_1 - \mathbf{r}_2||^2},
\end{equation}

incluindo, agora, no último termo, efeitos de maré devido à companheira. A equação de movimento para um elemento da estrela, nessa situação, é

\begin{equation}
    \label{eq:rochemotion}
    \ddot{\mathbf{r}}_1 = -\frac{1}{\rho}\nabla P - \nabla U_{ef} - 2\mathbf{\omega}\times\dot{\mathbf{r}}_1
\end{equation}

onde $\rho$, a densidade volumar de massa da estrela, e $P$, a pressão do material na estrela, descrevem a hidrodinâmica do material estelar devido à rotação do corpo. Caso a velocidade angular de rotação da estrela $\Omega_1$ e a velocidade angular orbital do binário $\omega$ sejam iguais, não há movimento do elemento de massa no nosso referencial, reduzindo a \ref{eq:rochemotion} ao equilíbrio hidrostático

\begin{equation}
    \label{eq:staticroche}
    \frac{1}{\rho}\nabla P = -\nabla U_{eff},
\end{equation}

donde se conclui que a densidade e pressão, e portanto os formatos possíveis da estrela, dependem unicamente do potencial, sendo dados pelas curvas equipotenciais.

Essa situação, que é o caso estático do binário de estrelas, admite, para órbitas não ecêntricas, pontos de Lagrange idênticos aos já descritos. O L1, por estar localizado entre as estrelas, descreve uma curva equipotencial limítrofe onde elas se tocam em um ponto, delimitando as regiões de captura gravitacional do material estelar, denominados \textit{lóbulos de Roche}, sendo que as regiões aproximadamente esféricas mais internas a cada objeto que caracterizam regiões de domínio gravitacionais são suas \textit{esferas de Hill}. Quando seus formatos excedem seus lóbulos de Roche, as estrelas passam a transferir massa entre si.

\subsection{Caso quasi-estático}

Mesmo em regimes fora das aproximações adotadas, com binários assíncronos e ecêntricos, é possível obter resultados bastante próximos para os pontos de Lagrange em certas situações. Com efeito, tomando o período do movimento interno de material, agora oscilatório devido à assíncronia,

\begin{equation}
    T_{mare} = \frac{2\pi}{|\omega - \Omega_1|}
\end{equation}

podemos definir a condição para um regime \textit{quasi-estático} do binário, no qual a \ref{eq:staticroche} vale instantaneamente com boa aproximação, por $T_{mare} << T_{dinamico} = \sqrt{\frac{R_1^3}{2Gm_1}}$, onde $T_{dinamico} $, a escala dinâmica da estrela, mede a escala de tempo em que a estrela expandiria ou contrairia numa situação em que \label{eq:staticroche} não vale.

Sob essas hipóteses, é possível demonstrar \cite{nonsyncbin} que há pontos estacionários do potencial com $y = 0$ que são soluções da equação,

\begin{equation}
    \frac{m_1}{m_2}\frac{xD^2}{|x|^3}+\frac{\frac{x}{D}-1}{|\frac{x}{D}-1|^3}-\frac{x}{D}\left(1+\frac{m_1}{m_2}\right)\frac{f^2(1+e)^4}{(1+e\cos(\theta - \theta_0))^3}+1=0
\end{equation}

e são análogos aos pontos L1, L2 e L3 do problema de três corpos circular planar. Os pontos triangulares L4 e L5 também podem ser determinados rigorosamente nesse tipo de sistema, com

\begin{align}
    (x, y) = \left(\frac{D}{2}\left(\frac{m_1}{m_2(1 + m_1/m_2)\frac{f^2(1+e)^4}{(1+e\cos(\theta - \theta_0))^3} - 1}\right)^{2/3},\pm D\sqrt{\frac{x}{D}\left(2 - \frac{x}{D}\right)}\right)
\end{align}

e condições de estabilidade análogas ao caso circular.