\section{Gravitação e o problema de n-corpos}

Dois corpos massivos ocupando posições diferentes no espaço sofrem uma atração proporcional ao produto de suas massas e ao inverso do quadrado de sua distância mútua no sentido que os une. Isso é descrito pelo potencial,

\begin{equation}
    \label{eq:gravitacaouniversal}
    U(\mathbf{r}_1 - \mathbf{r}_2) = -\frac{Gm_1m_2}{||\mathbf{r}_1 - \mathbf{r}_2||}
\end{equation}

Mas o espaço é homogêneo, e o movimento do sistema pode depender apenas das posições relativas dos objetos. Isso motiva que se defina

\begin{align}
    \mathbf{r} &= \mathbf{r}_1 - \mathbf{r}_2 && \mu = \frac{m_1m_2}{m_1 + m_2}
\end{align}

e as equações de movimento se reduzem a

\begin{equation}
    \mu \ddot{\mathbf{r}} = -\frac{\partial U}{\partial \mathbf{r}}
\end{equation}

e como

\begin{align}
    \mathbf{r}_1 &= \mathbf{r}_{CM} + \frac{m_2}{m_1 + m_2}\mathbf{r} && \mathbf{r}_2 &= \mathbf{r}_{CM} - \frac{m_1}{m_1 + m_2}\mathbf{r}
\end{align}

conclui-se que num referencial baricêntrico o problema de dois corpos é simplesmente o problema de força central, que é resolvido, pela mudança de variável $u = \frac{1}{r}$ num sistema de coordenadas polar perpendicular ao momento angular $\mathbf{L}$ (que é conservado), através da equação de Binet,

\begin{equation}
    \label{eq:binet}
    \frac{d^2u}{d\theta^2} + u = -\frac{\mu}{\mathbf{L}^2}\frac{d}{du}\left( U + \frac{\mathbf{L}^2u^2}{2\mu}\right).
\end{equation}

Assim, obtem-se a trajetória e, com algum esforço, a evolução temporal do sistema. Sistemas gravitacionais de mais de dois corpos são, contudo, impossíveis de se resolver analiticamente no caso geral. Num sistema de $n$ partículas interagindo entre si exclusivamente gravitacionalmente, a equação de movimento da $i$-ésima partícula é 

\begin{equation}
    \ddot{\mathbf{r}}_i=-\sum_{\substack{j=1\\j \neq i}}^{n} \frac{Gm_j(\mathbf{r}_j - \mathbf{r}_i)}{||\mathbf{r}_j - \mathbf{r}_i||^3}.
\end{equation} 

Trata-se de uma situação consideravelmente mais complexa que a de poucos corpos, uma vez que não há mais conservação de momento angular, a possibilidade de colisões gera singularidades nas soluções e o desacoplamento das equações não é possível. Apesar disso, é possível analisar e resolver sistemas de $n$-corpos, especialmente na presença de restrições e aproximações adequadas.

Os pontos de Lagrange emergem naturalmente na análise de um problema de três corpos como uma extensão do problema de Kepler: Sendo $a$ a distância entre dois objetos de massas $m_1$ e $m_2$, tomamos um referencial baricêntrico que gira com velocidade angular constante $\omega$ igual à orbital

\begin{equation}
    \omega^2 = \frac{G(m_1 + m_2)}{2\pi a^3}
\end{equation}

Caso um corpo estacionário de massa $m$ seja inserido nesse sistema, preservando coplanaridade, observar-se-á agindo sobre ele uma força centrífuga (e, caso se movimentasse, também uma força de Coriolis. Garantidamente não há força de Euler nesse tratamento), de forma que o potencial efetivo se torna

\begin{equation}
    U_{eff}(\mathbf{r}) = - \frac{Gm_i}{||\mathbf{r} - \mathbf{r}_1||} - \frac{Gm_i}{||\mathbf{r} - \mathbf{r}_2||} - \frac{1}{2}\mathbf{r}^2\omega^2
\end{equation}

e escolhendo um ponto no espaço $\mathbf{r_0} \neq \mathbf{r}_i$, observamos que o conjunto

\begin{align} 
    \Lambda &= \{\mathbf{r} | U_{eff}(\mathbf{r}) \geq U_{eff}(\mathbf{r_0}) - \epsilon\} && \epsilon > 0
\end{align}

é compacto, e portanto a função atinge nele um máximo $\mathbf{r_M}$. $\mathbf{r_M}$ está necessariamente contido no interior de $\Lambda$, já que $U_{eff}(\mathbf{r_M}) \geq U_{eff}(\mathbf{r_0})$, e pela diferenciabilidade de $U_{eff}$ o ponto é estacionário: um objeto de massa negligível nesse ponto realizará movimento circular uniforme ao redor do centro de massa do sistema.

Para fins práticos, entretanto, existência é insatisfatória, mas motiva o cálculo desses pontos, dentre outros pontos estacionários do potencial efetivo, que serão justamente os pontos de Lagrange.

\section{Dedução dos pontos de Lagrange}

% TODO Dedução os pontos L1, L2, L3, L4 e L5. Estabilidade dos pontos de Lagrange, valor de Gascheau.
% https://map.gsfc.nasa.gov/ContentMedia/lagrange.pdf

\section{Generalizações}
% TODO Discutir resultados recentes que generalizam a noção de ponto de Lagrange.
% http://iopscience.iop.org/article/10.1086/513736/meta
% http://iopscience.iop.org/article/10.1086/423214/meta
