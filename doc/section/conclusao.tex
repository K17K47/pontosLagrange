O presente trabalho objetivava apresentar o conceito de Pontos de Lagrange, bem como explicitar a teoria envolvida para a compreensão desse fenômeno, amplamente discutido no contexto da Mecânica Clássica. Para tal, foram utilizadas diversas ferramentas, como simulações computacionais, que coincidiram com as previsões teóricas (ratificando a instabilidade e a estabilidade dos pontos, de acordo com o esperado pelos autovalores) que foram obtidas por formalizações matemáticas, que suportaram o estudo dos mecanismos pelos quais são descritas interações entre três corpos, sob determinadas condições. Os problemas que circunscrevem esse fenômeno, por muito tempo, se apresentaram como  paradigmas da física teórica, justificando, portanto, a realização dessa obra.
